% Dokumentart auswählen: pa1, pa2, ba
\newcommand{\dokumentart}{pa2}


% Deckblatt Konfiguration
\newcommand{\projecttitle}{Zwischen Tradition und Innovation: Digitalisierung im deutschen Mittelstand}

\newcommand{\degreeprogram}{Studiengang Wirtschaftsinformatik}
\newcommand{\faculty}{in der Fakultät Wirtschaft}
\newcommand{\university}{an der Duale Hochschule Baden-Württemberg}
\newcommand{\universitycity}{Standort Heidenheim}

\newcommand{\studentname}{Max Mustermann}
\newcommand{\studentaddress}{Musterstraße 1}
\newcommand{\studentcity}{99999 Musterstadt}
\newcommand{\studentid}{1234567}

\newcommand{\companyname}{Beispiel GmbH}
\newcommand{\companyaddress}{Musterstraße 1}
\newcommand{\companycity}{99999 Musterstadt}

\newcommand{\semester}{2. Semester}
\newcommand{\tutorcompany}{Max Mustermann}
\newcommand{\tutoruniversity}{Erika Müller}

\newcommand{\signaturelocation}{Exampletown}
\newcommand{\signaturename}{John Doe}
\newcommand{\signaturefunction}{Lorem Ipsum Officer}


% Relevant für Bachelorarbeit
\newcommand{\degree}{Bachelor of Science}

% Ehrenwörtliche Erklärung
\newcommand{\wordcount}{9999}

% ----------------------------------------------------------------------------------------------------------

% Hilfsbefehl für Linien mit Text (z.B. Unterschrift)
\newcommand{\linedEntry}[3][0.3cm]{
  \makebox[#2]{\hrulefill}
  \hspace*{-#2}
  \makebox[#2][l]{\hspace*{#1}\raisebox{0.9ex}[0pt][0pt]{#3}}
}

% XeLaTeX: Unicode & Systemschriften
\RequirePackage[ngerman]{babel}
\usepackage{microtype}
\usepackage{fontspec}
\setmainfont{Arial}

% Seitenlayout
\usepackage[a4paper, left=3cm, right=3cm, top=2cm, bottom=2cm]{geometry}

% Absatz und Zeilenabstand
\usepackage{setspace}
\onehalfspacing         % 1,5-zeiliger Abstand
\setlength{\parskip}{6pt}   % 6pt Abstand zwischen Absätzen
\setlength{\parindent}{0pt} % Kein Einzug
\setlength{\skip\footins}{20pt}
\usepackage[
	bottom,
	hang,
	stable
]{footmisc}
\setlength{\footnotemargin}{0.5cm}

\usepackage[none]{hyphenat}
\setlength{\emergencystretch}{\maxdimen}

% Blocksatz
\usepackage{ragged2e}
\justifying

% Fußnoten: kleiner und einzeilig
\usepackage[bottom]{footmisc}
\renewcommand{\footnotesize}{\fontsize{10pt}{12pt}\selectfont} % 2pt kleiner als 12pt

% Nummerierung römisch für Verzeichnisse, arabisch für Textteil
\usepackage{fancyhdr}
\usepackage{etoolbox}

% Literaturverzeichnis sichtbar im Inhaltsverzeichnis
\usepackage[nottoc]{tocbibind}

% Links klickbar + saubere URL-Umbrüche (xurl optional)
\usepackage{xurl}
\usepackage[hidelinks]{hyperref}

% Bilder und Tabellen
\usepackage[figure,table]{totalcount}
\usepackage{tabularx}
\usepackage{graphicx}
\usepackage{caption}
\setlength{\belowcaptionskip}{-15pt}
\captionsetup[table]{
  name=Tab.,
  labelfont={bf},
  font={footnotesize}
}
\captionsetup[figure]{
  name=Abb.,
  labelfont={bf},
  font={footnotesize}
}
\usepackage{float}
\usepackage{longtable}
\usepackage{booktabs}
\usepackage{amsmath}

% Deckblatt
\usepackage{xstring}
\usepackage[datesep={.}, useregional={numeric}]{datetime2}

% Unicode-Mappings für Striche
\usepackage{newunicodechar}
\newunicodechar{–}{--}
\newunicodechar{—}{---}

%Glossar
\usepackage[nopostdot,nonumberlist,acronym,toc,style=list]{glossaries}
\makeglossaries

%Literaturverzeichnis & Zitation
\RequirePackage{csquotes}
\usepackage{bib_design_setup}
\addbibresource{literatur.bib}
\AtBeginBibliography{\RaggedRight}
